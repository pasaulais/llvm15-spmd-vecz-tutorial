%%%%%%%%%%%%%%%%%%%%%%%%%%%%%%%%%%%%%%%%%%%%%%%%%%%%%%%%%%%%%%%%%%%%%%%%%%%%%%%%

\begin{frame}{SIMD Width Detection}

\begin{itemize}
    \item Basic process
    \begin{itemize}
        \item Visit \varying{varying} nodes
        \item Record width $W$ of widest type used
        \item Given vector register width $V$, $N = \frac{V}{W}$
    \end{itemize}
    \item Improve analysis using register pressure information
    \begin{itemize}
        \item Max register usage $< \frac{1}{p}$? Multiply $N$ by $p$
        \item Result in $p$ times the number of native vector operations
    \end{itemize}
    \item Use a cost model
    \begin{itemize}
        \item Your target may only support some vector operations on specific widths
    \end{itemize}
\end{itemize}

\end{frame}

%%%%%%%%%%%%%%%%%%%%%%%%%%%%%%%%%%%%%%%%%%%%%%%%%%%%%%%%%%%%%%%%%%%%%%%%%%%%%%%%

\begin{frame}{Instantiation}

\begin{itemize}
    \item Not all instructions can be packetized
    \begin{itemize}
        \item External function with side-effcts (e.g. \texttt{printf})
        \item Atomic builtins
    \end{itemize}
    \item Solution: instantiate the instruction for all lanes ($i \in [0;N)$)
    \begin{itemize}
        \item Duplicate scalar instructions $N$ times
        %\item Replace $tid$ with \texttt{$tid + i$} (e.g. calls to \texttt{get\_global\_id}, \texttt{get\_local\_id})
        \item Replace $tid$ with \texttt{$tid + i$} (e.g. calls to \texttt{get\_global\_id})
        \item Need to extract packetized operands $N$ times %(using \texttt{extractelement})
    \end{itemize}
    \item Happens during the packetization stage
    \item Can be an alternative to scalarization
    \begin{itemize}
        \item Instantiation then becomes a stage of its own
        \item Analysis determines when to instatiate, when to scalarize
    \end{itemize}
\end{itemize}

\end{frame}

%%%%%%%%%%%%%%%%%%%%%%%%%%%%%%%%%%%%%%%%%%%%%%%%%%%%%%%%%%%%%%%%%%%%%%%%%%%%%%%%

\begin{frame}{Packetizing Builtin Function Calls}

\begin{itemize}
    \item Requires a 'builtin function database'
    \begin{itemize}
        \item Which functions are builtins
        \item Argument types, unless encoded in the mangled function name (e.g. OpenCL)
        \item Properties (e.g. returns an \varying{item ID}, has side-effects, pointer return, etc)
    \end{itemize}
    \item Map scalar builtin to vector equivalent (e.g. \prototype{tan(float)} to \prototype{tan(float4)})
    \begin{itemize}
        \item And the other way around, for scalarization
        \item Assumes the equivalent builtin is already packetized
    \end{itemize}
    \item Simply create call to vector equivalent with packetized operands
    \item However not all OpenCL builtins have vector equivalents (e.g. \prototype{float dot(float4)})
    \begin{itemize}
        \item Can inline these builtins before scalarization
    \end{itemize}
\end{itemize}

\end{frame}

%%%%%%%%%%%%%%%%%%%%%%%%%%%%%%%%%%%%%%%%%%%%%%%%%%%%%%%%%%%%%%%%%%%%%%%%%%%%%%%%

\begin{frame}[fragile]{Packetizing Builtin Functions}

\begin{itemize}
    \item Clone builtin function, updating signature (vector return value, arguments)
    \item Generate argument placeholders (see example below), replacing all uses
    \item UVA treats arguments as roots
    \item \texttt{ret} instructions are leaves
    \item Packetizing a placeholder replaces \texttt{extractelement} by the argument
\end{itemize}

IR after cloning, before packetization:

\begin{codebox}[commandchars=\\\[\]]
define <4 x i32> @__v4__Z7isequalff(<4 x float> \varying[%x], <4 x float> \varying[%y]) {
entry:
  \varying[%placeholder_x] = extractelement <4 x float> \varying[%x], i32 0
  \varying[%placeholder_y] = extractelement <4 x float> \varying[%y], i32 0
  \varying[%cmp.i] = fcmp oeq float \varying[%placeholder_x], \varying[%placeholder_y]
  \varying[%conv.i] = zext i1 \varying[%cmp.i] to i32
  ret i32 \varying[%conv.i]
}
\end{codebox}

\end{frame}

%%%%%%%%%%%%%%%%%%%%%%%%%%%%%%%%%%%%%%%%%%%%%%%%%%%%%%%%%%%%%%%%%%%%%%%%%%%%%%%%

\begin{frame}{Packetizing User Functions (No Side-Effects)}

\begin{itemize}
    \item Vectorizer has no intrinsic knowledge of which arguments are \varying{varying}
    \begin{itemize}
        \item Need to analyze this for each call site
        \item Generate a packetized function for each combination
    \end{itemize}
    \item Arguments may also be \uniform{uniform} (e.g. arrays)
    \item Otherwise similar to packetizing builtins 
\end{itemize}

\end{frame}

%%%%%%%%%%%%%%%%%%%%%%%%%%%%%%%%%%%%%%%%%%%%%%%%%%%%%%%%%%%%%%%%%%%%%%%%%%%%%%%%

\begin{frame}{Packetizing User Functions (Side-Effects)}

\begin{itemize}
    \item The vectorized function needs to take an extra mask argument, $m_{entry}$
    \begin{itemize}
        \item Determines which lanes are enabled when entering the function
    \end{itemize}
    \item When applying masks, pass $m_{B}$ to function calls
    \begin{itemize}
        \item Where $B$ is the block where the call instruction is
    \end{itemize}
    \item Might be simpler to just inline such calls
\end{itemize}

\end{frame}

%%%%%%%%%%%%%%%%%%%%%%%%%%%%%%%%%%%%%%%%%%%%%%%%%%%%%%%%%%%%%%%%%%%%%%%%%%%%%%%%

\begin{frame}[fragile]{CFG Specialization}

\begin{itemize}
    \item Duplicate part of the CFG
    \begin{itemize}
        \item With the assumption that all lanes are enabled
        \item Avoids CFG conversion and predication for the specialized part
        \item Increases code size
    \end{itemize}
    \item Need to generate an extra branch (guard) to specialized code
    \begin{itemize}
        \item e.g. \texttt{'b i1 $all(m_{A->B})$, label $\%B_{spec}$, label $\%B$'}
    \end{itemize}
\end{itemize}

\begin{codebox}[commandchars=\\\[\]]
kernel void convolution(float \uniform[*src], float \uniform[*dst]) {
  int \varying[x] = \varying[get_global_id(0)];
  int \uniform[width] = \uniform[get_global_size(0)];
  float sum = \uniform[0.0f];
  if ((\varying[x] >= \uniform[FILTER_SIZE]) && (\varying[x] < (\uniform[width] - \uniform[FILTER_SIZE]))) {
    /* Loop that computes sum, using an \uniform[uniform] condition */
  }
  \uniform[dst]\idx[\varying[x]] = \varying[sum];
}
\end{codebox}

\end{frame}

%%%%%%%%%%%%%%%%%%%%%%%%%%%%%%%%%%%%%%%%%%%%%%%%%%%%%%%%%%%%%%%%%%%%%%%%%%%%%%%%

\begin{frame}{CFG Conversion: Single Lane}

\begin{itemize}
    \item Branch always taken by a single lane
    \begin{itemize}
        \item e.g. \texttt{'if (tid == 0) \{ /* write back result */ \}'}
        \item Often used with reductions
    \end{itemize}
    \item No need for CFG conversion
    \begin{itemize}
        \item Keep the conditional branch
    \end{itemize}
    \item No need for packetization
\end{itemize}

\end{frame}

%%%%%%%%%%%%%%%%%%%%%%%%%%%%%%%%%%%%%%%%%%%%%%%%%%%%%%%%%%%%%%%%%%%%%%%%%%%%%%%%

\begin{frame}{Interleaved Memory Optimizations}

\begin{itemize}
    \item Scalarizing vector loads and stores results in many interleaved loads and stores
    \item Most targets do not support this efficiently
    \begin{itemize}
        \item Resulting in even more scalar loads, stores, vector extractions and insertions
    \end{itemize}
    \item Grouping these instructions often helps
    \begin{itemize}
        \item ARM supports \texttt{vld.[2-4]} and \texttt{vst.[2-4]} for some vector types
        \item Can replace $n$-group with $n$ memory operations and $n \times n$-transposition 
    \end{itemize}
    \item To find groups, look for a common base pointer and increasing offsets
\end{itemize}


\end{frame}

%%%%%%%%%%%%%%%%%%%%%%%%%%%%%%%%%%%%%%%%%%%%%%%%%%%%%%%%%%%%%%%%%%%%%%%%%%%%%%%%

\begin{frame}{SoA to AoS Conversion}

\end{frame}
