%%%%%%%%%%%%%%%%%%%%%%%%%%%%%%%%%%%%%%%%%%%%%%%%%%%%%%%%%%%%%%%%%%%%%%%%%%%%%%%%

\talksection{Overview}

\begin{frame}{SPMD Vectorizer}

\begin{itemize}
    \item Is used to vectorize a SPMD program's entry point function
    \item Given a function $F$ and vectorization factor $N$, produces a function $VF_N$
    \begin{itemize}
        \item Calling $VF_N$ is like calling $F$, but $N$ times (on consecutive work-items)
        \item $F$ and $VF_N$ have the same signature
    \end{itemize}
    
    \item The original function can be preserved or not
    \begin{itemize}
        \item In-place: vectorize the original function, no need for cloning
        \item Transform cloned function: allow vectorization to fail
    \end{itemize}
    
    \item Vectorization may be allowed to fail
    \begin{itemize}
        \item On failure, the original function can be used 
    \end{itemize}
\end{itemize}

\end{frame}

%%%%%%%%%%%%%%%%%%%%%%%%%%%%%%%%%%%%%%%%%%%%%%%%%%%%%%%%%%%%%%%%%%%%%%%%%%%%%%%%

\begin{frame}{Structure}

\begin{itemize}
    \item Pipeline structure
    \begin{itemize}
        \item Function to vectorize is repeatedly transformed by different stages
        \item Stages are independent from each other
        \item Each stage consists of one or more IR passes
        \item Most stages require some analysis
    \end{itemize}
    
    \item The LLVM pass manager (both new and original) is a good fit here
    \begin{itemize}
        \item Most passes work at the function level
        \item Access to the module is sometimes needed
    \end{itemize}
\end{itemize}

\end{frame}

%%%%%%%%%%%%%%%%%%%%%%%%%%%%%%%%%%%%%%%%%%%%%%%%%%%%%%%%%%%%%%%%%%%%%%%%%%%%%%%%

\begin{frame}[c]{Stages}

[Stage diagram with stages:\\ preparation, CFG conversion, scalarization, packetization, optimizations]

\end{frame}

%%%%%%%%%%%%%%%%%%%%%%%%%%%%%%%%%%%%%%%%%%%%%%%%%%%%%%%%%%%%%%%%%%%%%%%%%%%%%%%%

\begin{frame}{Analyses}

\begin{itemize}
    \item Capture information about the IR to vectorize
    \item May be invalidated after IR transformations
    \item May depend on other analyses
\end{itemize}

\end{frame}

%%%%%%%%%%%%%%%%%%%%%%%%%%%%%%%%%%%%%%%%%%%%%%%%%%%%%%%%%%%%%%%%%%%%%%%%%%%%%%%%

\begin{frame}{Analyses}

\begin{itemize}
    \item Uniform Value Analysis
    \begin{itemize}
        \item Marks values as either \uniform{uniform} or \varying{varying}
    \end{itemize}

    \item Control Flow Analysis
    \begin{itemize}
        \item Determines which basic blocks are \varying{divergent}
        \item Builds a Control Dependency Graph
    \end{itemize}
    
    \item SIMD Width Analysis
    \begin{itemize}
        \item Chooses a 'good' width $N$ based on register/instruction usage
    \end{itemize}
\end{itemize}

\end{frame}

%%%%%%%%%%%%%%%%%%%%%%%%%%%%%%%%%%%%%%%%%%%%%%%%%%%%%%%%%%%%%%%%%%%%%%%%%%%%%%%%

\begin{frame}{IR vs MI: Advantages and Drawbacks}

\begin{itemize}
    \item IR Level

    \item MI level
    \begin{itemize}
        \item (mention Marcello's talk)
    \end{itemize}
    
\end{itemize}

\end{frame}

%%%%%%%%%%%%%%%%%%%%%%%%%%%%%%%%%%%%%%%%%%%%%%%%%%%%%%%%%%%%%%%%%%%%%%%%%%%%%%%%

%\begin{frame}{Implementation strategy}
%
%% TODO: Skip slide?
%\begin{itemize}
%    \item Create test kernels
%    \begin{itemize}
%        \item Start with very simple kernels (e.g. copy buffer, add two buffers)
%        \item Gradually add more features (e.g. non-sequential memory accesses, vector instructions, etc)
%    \end{itemize}
%
%    
%    \item Suggested implementation order
%    \begin{itemize}
%        \item Preparation and packetization first (required for simplest kernels)
%        \item Then easier features: builtins, memory addressing, scalarization, instantiation
%        \item More complex features last: control flow, optimizations
%    \end{itemize}
%\end{itemize}
%
%\end{frame}

