% This presentation is based on a Beamer theme from Seth Brown, distributed
% under the following license:
%
% ----------------------------------------------------------------------------
% This program can be redistributed and/or modified under the terms
% of the GNU Public License, version 3.
%
% Seth Brown, Ph.D.
% sethbrown@drbunsen.org

\documentclass[t]{beamer}

\usepackage{hyperref} % urls
\usepackage{graphicx} % graphics
\usepackage[utf8x]{inputenc}
\usepackage{fancyvrb}
\usepackage{tcolorbox}
\usepackage{animate}

% suppress navigation bar
\beamertemplatenavigationsymbolsempty

\mode<presentation>
{
  \usetheme{codeplay}
  \setbeamercovered{transparent}
  %\setbeamertemplate{itemize item}[circle]
  \setbeamertemplate{itemize item}{\small\raise0.8pt\hbox{$\bullet$}}
  %\setbeamertemplate{itemize subitem}{\tiny\raise1.5pt\hbox{\donotcoloroutermaths$\blacktriangleright$}}
  \setbeamertemplate{itemize subitem}{\tiny\raise1.5pt\hbox{$\blacktriangleright$}}
  \setbeamertemplate{section in toc shaded}[default][100]
  \setbeamertemplate{subsection in toc shaded}[default][100]

\setbeamertemplate{subsection in toc}{%
\leavevmode\leftskip=4.00ex%
  \llap{\raisebox{0.15ex}{\textcolor{structure}{\small$\bullet$}}\kern1ex}%
  \inserttocsubsection\par%
}
}

% set fonts
\usepackage{fontspec}
\setsansfont{Calibri}
\setmonofont{Consolas}
\setbeamerfont{frametitle}{size=\LARGE,series=\bfseries}

% color definitions
\usepackage{color}
\definecolor{uiwhite}{RGB}{255, 255, 255}
\definecolor{uidarkgray}{RGB}{24,24,25}
\definecolor{uigray}{RGB}{51,51,51}
\definecolor{uilightgray}{RGB}{123,123,123}
\definecolor{uicyan}{RGB}{0,255,255}
\definecolor{uipink}{RGB}{255,153,255}
\definecolor{uipalegreen}{RGB}{153,255,153}
\definecolor{uiorange}{RGB}{237,175,28}
\hypersetup{colorlinks,urlcolor=uicyan,linkcolor=uiwhite}

% title slide definition
\title{Creating an SPMD Vectorizer for OpenCL with LLVM}
\subtitle{LLVM 2015 tutorial}
\author{Pierre-André Saulais}
\institute{Codeplay Software \\ @codeplaysoft}

\date{\today}

% Code box formatting
\fvset{fontsize=\scriptsize,tabsize=4}
\tcbset{colback=uigray,colframe=uilightgray,colupper=uiwhite,
        left=1pt,right=1pt,top=1pt,bottom=1pt,arc=0pt,
        toprule=1pt,bottomrule=1pt,leftrule=1pt,rightrule=1pt}

\newenvironment{codebox}
 {\VerbatimEnvironment
  \begin{tcolorbox}%
  \begin{Verbatim}}
 {\end{Verbatim}\end{tcolorbox}}

% Code caption that comes after the code box.
\newcommand{\codecaption}[2][-4.4ex]{%
  \vspace{#1}
  \hfill\mbox{\footnotesize{\textcolor{uicyan}{#2}}\hspace{0.5em}}
  \vspace*{2ex}}

\newcommand{\sourcebox}[3][firstline=1]{%
  \begin{tcolorbox}\VerbatimInput[#1]{#2}\end{tcolorbox}
  \codecaption{#3}}

\newcommand{\examplebox}[2][firstline=1]{\sourcebox[#1]{examples/#2}{#2}}

\newcommand{\codeempha}[1]{\textcolor{uipink}{#1}}
\newcommand{\codeemphb}[1]{\textcolor{uicyan}{#1}}
\newcommand{\codeemphc}[1]{\textcolor{uipalegreen}{#1}}
\newcommand{\codeemphd}[1]{\textcolor{uiorange}{#1}}

\newcommand{\varying}[1]{\codeempha{#1}}
\newcommand{\uniform}[1]{\codeemphb{#1}}
\newcommand{\idx}[1]{[#1]}

%\newcommand{\penguin}[0]{\includegraphics[width = 1em]{images/penguin.png}}

\newcommand{\penguin}[0]{\animategraphics[loop,autoplay, width = 1em]{12}{images/penguin-}{0}{19}\hspace{0.1em}}

% Create a slide for a part.
\newcommand{\talkpart}[2]{%
\section{Part #1: #2}

\begin{frame}[c]{Part #1}

\vspace{1cm}
\centerline{\LARGE{#2}}

\end{frame}}

% Create a slide for a part section.
\newcommand{\talksection}[1]{%
\subsection{#1}
\begin{frame}{\insertsectionhead}
\vspace{2ex}
\begin{NoHyper}
\tableofcontents[sectionstyle=hide,subsectionstyle=show/shaded/hide]
\end{NoHyper}
\end{frame}}

%%%%%%%%%%%%%%%%%%%%%%%%%%%%%%%%%%%%%%%%%%%%%%%%%%%%%%%%%%%%%%%%%%%%%%%%%%%%%%%%

\begin{document}

\setbeamertemplate{background}
{\includegraphics[width=\paperwidth,height=\paperheight]{dark_background_title.png}}
\setbeamertemplate{footline}[default]

\begin{frame}
  \vspace{4ex}
  \titlepage
\end{frame}

%%%%%%%%%%%%%%%%%%%%%%%%%%%%%%%%%%%%%%%%%%%%%%%%%%%%%%%%%%%%%%%%%%%%%%%%%%%%%%%%

% Set the background for the rest of the slides.
%\setbeamertemplate{background}
% {\includegraphics[width=\paperwidth,height=\paperheight]{dark_background.png}}
\setbeamertemplate{background}{}
\setbeamertemplate{footline}[codeplaytheme]

\section*{Introduction}

\begin{frame}{What this tutorial is about}

\begin{itemize}
    \item Vectorizing
    \begin{itemize}
        \item Transform whole functions using LLVM
        \item "Horizontal" vectorization
    \end{itemize}  
    \item SPMD programs
    \begin{itemize}
        \item Data-parallel execution model
        \item Used with compute frameworks like OpenCL and CUDA
    \end{itemize}
    \item for CPU-like processors, i.e. each core has
    \begin{itemize}
        \item A program counter
        \item Vector SIMD unit(s)
    \end{itemize}
\end{itemize}

\end{frame}

%%%%%%%%%%%%%%%%%%%%%%%%%%%%%%%%%%%%%%%%%%%%%%%%%%%%%%%%%%%%%%%%%%%%%%%%%%%%%%%%

\begin{frame}{Overview}
\tableofcontents
\end{frame}

%%%%%%%%%%%%%%%%%%%%%%%%%%%%%%%%%%%%%%%%%%%%%%%%%%%%%%%%%%%%%%%%%%%%%%%%%%%%%%%%

%\talkpart{1}{Background}
%%%%%%%%%%%%%%%%%%%%%%%%%%%%%%%%%%%%%%%%%%%%%%%%%%%%%%%%%%%%%%%%%%%%%%%%%%%%%%%%%

\talksection{SPMD Execution Model}

\begin{frame}{SPMD Execution Model}

\begin{itemize}
    \item Data-parallel
    \begin{itemize}
        \item Work needs to be divided
    \end{itemize}
    
    \item Single program, scalar with implicit SIMD execution
    
    \item Multiple instances running in parallel
    \begin{itemize}
        \item Each instance working on different data
        \item On GPU, SIMD execution in lockstep
        \item On CPU, sequential execution within a core (naive approach)
    \end{itemize}
\end{itemize}

\begin{itemize}
    \item Simplistic example:
    \begin{itemize}
        \item Massive Online Course
        \item Compute the overall grade (GPA) of millions of students
        \item GPA is the weighted average of several grades
    \end{itemize}
\end{itemize}

\end{frame}

%% Kernel function has no return value, takes in buffers (arrays)
%% Vectorization does not change the signature for kernels
%% Can also vectorize "normal" functions, but this results in signature changes

%%%%%%%%%%%%%%%%%%%%%%%%%%%%%%%%%%%%%%%%%%%%%%%%%%%%%%%%%%%%%%%%%%%%%%%%%%%%%%%%

\begin{frame}{Division of Work}

\begin{itemize}
    \item Work-item:
    \begin{itemize}
        \item Unit of work
        \item One instance of a program
    \end{itemize}
    \item Work-items executed in parallel by Execution Units (threads)
    \item Dimensions
    \begin{itemize}
        \item 1D (array shape)
        \item 2D (grid shape)
        \item ...
    \end{itemize}
\end{itemize}

[graph]

\end{frame}

%%%%%%%%%%%%%%%%%%%%%%%%%%%%%%%%%%%%%%%%%%%%%%%%%%%%%%%%%%%%%%%%%%%%%%%%%%%%%%%%

\begin{frame}[fragile]{Single Program}

\begin{itemize}
    \item Kernel function
    \begin{itemize}
        \item Entry point for the computation
    \end{itemize}
    \item Executed once per work-item
    \begin{itemize}
        \item As if there was a loop around it
        \item Access to the iteration counter using \texttt{get\_global\_id(0)}
    \end{itemize}
\end{itemize}

\begin{codebox}
kernel void calc_gpa(global *float result, global *int grades, global *float weights,
                     int num_grades, int num_students) {
    int student_id = get_global_id(0);
    float gpa = 0.0;
    for (int i = 0; i < num_grades; i++) {
        int grade = grades[(i * num_students) + student_id];
        float weight = weights[i];
        gpa += (grade * weight);
    }
    result[student_id] = gpa;
}
\end{codebox}

\end{frame}

%%%%%%%%%%%%%%%%%%%%%%%%%%%%%%%%%%%%%%%%%%%%%%%%%%%%%%%%%%%%%%%%%%%%%%%%%%%%%%%%

\talksection{Vectorization}
\begin{frame}{Why Vectorize?}

\begin{itemize}
    \item Many executions units each executing one instance of a single program 
    \begin{itemize}
        \item Works well on GPU (many hardware EUs)
        \item Not so much on CPU (very few hardware EUs)
        \item CPU has to execute many work-items sequentially
    \end{itemize}
    \item Speed up this sequential computation using SIMD units
    \begin{itemize}
        \item Vertical Vectorization
        \item Horizontal Vectorization
    \end{itemize}
\end{itemize}

\end{frame}

%%%%%%%%%%%%%%%%%%%%%%%%%%%%%%%%%%%%%%%%%%%%%%%%%%%%%%%%%%%%%%%%%%%%%%%%%%%%%%%%

\begin{frame}{Vertical Vectorization}

\begin{itemize}
    \item Within a single work-item
    \begin{itemize}
        \item e.g. loops within a kernel
        \item In our example, the weighted average computation
        \end{itemize}
    \item Using the LLVM Loop Vectorizer
    \item However, not all kernels contain loops
\end{itemize}

[graph of work-items with vertical loops]

\end{frame}

%%%%%%%%%%%%%%%%%%%%%%%%%%%%%%%%%%%%%%%%%%%%%%%%%%%%%%%%%%%%%%%%%%%%%%%%%%%%%%%%

\begin{frame}{Horizontal Vectorization}

\begin{itemize}
    \item Across work-items
    \begin{itemize}
        \item Compute multiple work-items at the same time
        \item Take advantage of the execution model (single program, multiple data)
    \end{itemize}
\end{itemize}

[graph of work-items with horizontal arrows]

\end{frame}

%%%%%%%%%%%%%%%%%%%%%%%%%%%%%%%%%%%%%%%%%%%%%%%%%%%%%%%%%%%%%%%%%%%%%%%%%%%%%%%%

\begin{frame}{Vectorizer Comparison}

\begin{itemize}
    \item Loop Vectorizer
    \begin{itemize}
        \item Can be used for both vertical and horizontal vectorization
        \item Has to enforce dependencies between loop iterations
        \item Execution order is not specified, this is not needed
        \item Nested control-flow?
    \end{itemize}

    \item SLP Vectorizer
    \begin{itemize}
        \item Finds groups of similar scalar instructions (same opcode)
        \item Vertical vectorization
        \item Not all kernels contain this kind of code
    \end{itemize}

    \item SPMD Vectorizer
    \begin{itemize}
        \item Only supports horizontal vectorization
        \item Nested control-flow
        \item Not limited to a certain 'style' of code
    \end{itemize}
\end{itemize}

\end{frame}

%%%%%%%%%%%%%%%%%%%%%%%%%%%%%%%%%%%%%%%%%%%%%%%%%%%%%%%%%%%%%%%%%%%%%%%%%%%%%%%%

\begin{frame}{Glossary}

\begin{itemize}
    \item Work
    \begin{itemize}
        \item Work-item: unit of work to execute in parallel.
        \item Instance: State of one work-item.
        \item (SIMD) Lane: Execution of one instance, after vectorization.
        \end{itemize}
        
    \item Data
    \begin{itemize}
        \item Packet: Contains several values/instructions, one per SIMD lane. Corresponds to one value/instruction in the original kernel.
        \item \uniform{Uniform}: Packet where values are identical for all lanes.
        \item \varying{Varying}: Packet where values are not identical for all lanes.
        \end{itemize}

    \item Control Flow
    \begin{itemize}
        \item \uniform{Uniform}: Branch taken by all lanes.
        \item \varying{Divergent}: Branch taken by some lanes. Requires special handling (No SIMD branching)
    \end{itemize}
\end{itemize}

\end{frame}


\talkpart{2}{Implementing a SPMD Vectorizer}
%%%%%%%%%%%%%%%%%%%%%%%%%%%%%%%%%%%%%%%%%%%%%%%%%%%%%%%%%%%%%%%%%%%%%%%%%%%%%%%%

\talksection{Design}

\begin{frame}{Design}

\begin{itemize}
    \item Pipeline structure
    \begin{itemize}
        \item Function to vectorize is repeatedly transformed by different stages
        \item Each stage consists of one or more IR passes
        \item Most stages require some analysis
        \begin{itemize}
            \item May be run mulitple times as stages invalidate it
        \end{itemize}
    \end{itemize}
    
    \item The LLVM pass manager (both new and original) is a good fit here
    \begin{itemize}
        \item Most passes work at the function level
        \item Access to the module is sometimes needed
    \end{itemize}
    
    \item The original function can be preserved or not
    \begin{itemize}
        \item In-place: vectorize the original function, no need for cloning
        \item Work on cloned function: allow vectorization to fail
    \end{itemize}
\end{itemize}

\end{frame}

%%%%%%%%%%%%%%%%%%%%%%%%%%%%%%%%%%%%%%%%%%%%%%%%%%%%%%%%%%%%%%%%%%%%%%%%%%%%%%%%

\begin{frame}{Stages}

% Stage diagram with stages: preparation, CFG conversion, scalarization, packetization, optimizations

\end{frame}

%%%%%%%%%%%%%%%%%%%%%%%%%%%%%%%%%%%%%%%%%%%%%%%%%%%%%%%%%%%%%%%%%%%%%%%%%%%%%%%%

\begin{frame}{Analyses}

% Two important anlyses: UVA and CFG

\end{frame}

%%%%%%%%%%%%%%%%%%%%%%%%%%%%%%%%%%%%%%%%%%%%%%%%%%%%%%%%%%%%%%%%%%%%%%%%%%%%%%%%

\begin{frame}{Implementation strategy}

\begin{itemize}
    \item Create test kernels
    \begin{itemize}
        \item Start with very simple kernels (e.g. copy buffer, add two buffers)
        \item Gradually add more features (e.g. non-sequential memory accesses, vector instructions, etc)
        \item
    \end{itemize}

    
    \item Suggested implementation order
    \begin{itemize}
        \item Preparation and packetization first (required for simplest kernels)
        \item Then easier features: builtins, memory addressing, scalarization, instantiation
        \item More complex features last: control flow, optimizations
    \end{itemize}
\end{itemize}

\end{frame}

%%%%%%%%%%%%%%%%%%%%%%%%%%%%%%%%%%%%%%%%%%%%%%%%%%%%%%%%%%%%%%%%%%%%%%%%%%%%%%%%

\talksection{Stage: Packetization}

\begin{frame}{Packetization Overview}

\begin{itemize}
    \item Stage that does the actual vectorization
    \begin{itemize}
        \item Needs a vectorization factor $N$ (SIMD width)
        \item Calling the transformed function is like calling the original $N$ times
        \item Straightforward thanks to preparation from previous stages
    \end{itemize}
    
    \item This is done per-instruction, for the whole function
    \begin{itemize}
        \item Instructions that define a value: define $N$ values, one for each instance
        \item Instructions with side effects: perform side effects for each instance
    \end{itemize}
    
    \item Only varying instructions need packetization
    \begin{itemize}
        \item Uniform instructions can remain scalar, executed once per work-group
        \item Requires \textbf{Uniform Value Analysis} to know which instructions to vectorize
    \end{itemize}
    
\end{itemize}

\end{frame}

%%%%%%%%%%%%%%%%%%%%%%%%%%%%%%%%%%%%%%%%%%%%%%%%%%%%%%%%%%%%%%%%%%%%%%%%%%%%%%%%

\begin{frame}{Uniform Value Analysis}

\begin{itemize}
    \item Finds 'root' values
    \begin{itemize}
        \item Varying value that doesn't have any varying operand
        \item Example: \texttt{get\_global\_id(0)} has a different value for each isntance
    \end{itemize}
    \item Marks each IR value as 'uniform' or 'varying'
    \begin{itemize}
        \item All values start as 'uniform'
        \item Marking a value as 'varying' causes all users to also be marked 'varying'
        \item Marking is done recursively, starting with roots
        \item Values are marked before their users, to avoid cycles (phi nodes)
    \end{itemize}
\end{itemize}

\end{frame}

%%%%%%%%%%%%%%%%%%%%%%%%%%%%%%%%%%%%%%%%%%%%%%%%%%%%%%%%%%%%%%%%%%%%%%%%%%%%%%%%

\begin{frame}[fragile]{Uniform Value Analysis Example}

Example that combines uniform and varying values:

\begin{codebox}[commandchars=\\\[\]]
kernel void add_uniform(global int *\codeemphb[dst], global int *\codeemphb[src], int \codeemphb[alpha]) {
    int \codeempha[tid] = \codeempha[get_global_id](0);
    \codeemphb[dst]\idx[\codeempha[tid]] = \codeemphb[src]\idx[\codeempha[tid]] + (\codeemphb[alpha] + \codeemphb[1]);
}
\end{codebox}

\begin{codebox}[commandchars=\\\[\]]
define void @add_uniform(i32 addrspace(1)* \codeemphb[%dst], i32 addrspace(1)* \codeemphb[%src],
                         i32 \codeemphb[%alpha]) {
entry:
  \codeempha[%tid] = i32 \codeempha[@get_global_id(i32 0)]
  \codeempha[%arrayidx] = getelementptr inbounds i32 addrspace(1)* \codeemphb[%src], i32 \codeempha[%tid]
  \codeempha[%tmp] = load i32 addrspace(1)* \codeempha[%arrayidx], align 4
  \codeemphb[%add] = add nsw i32 \codeemphb[%alpha], \codeemphb[1]
  \codeempha[%add1] = add nsw i32 \codeemphb[%add], \codeempha[%tmp]
  \codeempha[%arrayidx2] = getelementptr inbounds i32 addrspace(1)* \codeemphb[%dst], i32 \codeempha[%tid]
  store i32 \codeempha[%add1], i32 addrspace(1)* \codeempha[%arrayidx2], align 4
  ret void
}
\end{codebox}

% Mix of uniform and varying values
\end{frame}

%%%%%%%%%%%%%%%%%%%%%%%%%%%%%%%%%%%%%%%%%%%%%%%%%%%%%%%%%%%%%%%%%%%%%%%%%%%%%%%%

\begin{frame}{Uniform Value Analysis Example}

% IR graph to show values, highlight uniform and varying ones

\end{frame}

%%%%%%%%%%%%%%%%%%%%%%%%%%%%%%%%%%%%%%%%%%%%%%%%%%%%%%%%%%%%%%%%%%%%%%%%%%%%%%%%

\begin{frame}{Packetization Process}

\begin{itemize}
    \item Start at leaves
    \item Packetize operands before instructions
    \item Broadcast uniform operands
    \item Cache packetized values to prevent duplication
\end{itemize}

\end{frame}

%%%%%%%%%%%%%%%%%%%%%%%%%%%%%%%%%%%%%%%%%%%%%%%%%%%%%%%%%%%%%%%%%%%%%%%%%%%%%%%%

\begin{frame}{Packetization Example}

\begin{itemize}
    \item How does it work?
\end{itemize}

\end{frame}

%%%%%%%%%%%%%%%%%%%%%%%%%%%%%%%%%%%%%%%%%%%%%%%%%%%%%%%%%%%%%%%%%%%%%%%%%%%%%%%%

\begin{frame}{Packetization: Phi nodes}

\begin{itemize}
    \item How to handle this special case
\end{itemize}

\end{frame}

%%%%%%%%%%%%%%%%%%%%%%%%%%%%%%%%%%%%%%%%%%%%%%%%%%%%%%%%%%%%%%%%%%%%%%%%%%%%%%%%

\begin{frame}{Memory addressing}

\begin{itemize}
    \item Uniform address -> scalar loads and stores
    \item Constant stride =1 -> vector laods and stores
    \item Constant stride >1 -> interleaved loads and stores
    \item Other -> gather loads, scatter stores
\end{itemize}

% Generate calls to internal builtins
% Internal builtins can be implemented for each target as supported

\end{frame}

%%%%%%%%%%%%%%%%%%%%%%%%%%%%%%%%%%%%%%%%%%%%%%%%%%%%%%%%%%%%%%%%%%%%%%%%%%%%%%%%

\talksection{Stage: Scalarization}

\begin{frame}{Scalarization Overview}

\begin{itemize}
    \item What does it do?
    \item Requires scalarization analysis
\end{itemize}

\end{frame}

%%%%%%%%%%%%%%%%%%%%%%%%%%%%%%%%%%%%%%%%%%%%%%%%%%%%%%%%%%%%%%%%%%%%%%%%%%%%%%%%

\begin{frame}{Scalarization Analysis}

\begin{itemize}
    \item Looks for vector instructions
    \begin{itemize}
        \item Leaves that define vector values, vector stores
        \item Vector extractions
        \item Vector -> scalar bitcasts
    \end{itemize}
    
\end{itemize}

\end{frame}

%%%%%%%%%%%%%%%%%%%%%%%%%%%%%%%%%%%%%%%%%%%%%%%%%%%%%%%%%%%%%%%%%%%%%%%%%%%%%%%%

\begin{frame}{Scalarization Process}


\end{frame}

%%%%%%%%%%%%%%%%%%%%%%%%%%%%%%%%%%%%%%%%%%%%%%%%%%%%%%%%%%%%%%%%%%%%%%%%%%%%%%%%

\begin{frame}{Scalarization Example}


\end{frame}

%%%%%%%%%%%%%%%%%%%%%%%%%%%%%%%%%%%%%%%%%%%%%%%%%%%%%%%%%%%%%%%%%%%%%%%%%%%%%%%%

\talksection{Stage: Control Flow Conversion}

\begin{frame}{Control Flow Conversion Overview}

\begin{itemize}
    \item What does it do?
    \item Why is it needed?
\end{itemize}

\end{frame}

%%%%%%%%%%%%%%%%%%%%%%%%%%%%%%%%%%%%%%%%%%%%%%%%%%%%%%%%%%%%%%%%%%%%%%%%%%%%%%%%

\begin{frame}{Control Flow Conversion: if}

\end{frame}

%%%%%%%%%%%%%%%%%%%%%%%%%%%%%%%%%%%%%%%%%%%%%%%%%%%%%%%%%%%%%%%%%%%%%%%%%%%%%%%%

\begin{frame}{Mask Generation}

\end{frame}

%%%%%%%%%%%%%%%%%%%%%%%%%%%%%%%%%%%%%%%%%%%%%%%%%%%%%%%%%%%%%%%%%%%%%%%%%%%%%%%%

\begin{frame}{Applying Masks}

\end{frame}

%%%%%%%%%%%%%%%%%%%%%%%%%%%%%%%%%%%%%%%%%%%%%%%%%%%%%%%%%%%%%%%%%%%%%%%%%%%%%%%%

\begin{frame}{Masked Memory Operations}

\end{frame}

%%%%%%%%%%%%%%%%%%%%%%%%%%%%%%%%%%%%%%%%%%%%%%%%%%%%%%%%%%%%%%%%%%%%%%%%%%%%%%%%

\begin{frame}{Phi Conversion}

\end{frame}

%%%%%%%%%%%%%%%%%%%%%%%%%%%%%%%%%%%%%%%%%%%%%%%%%%%%%%%%%%%%%%%%%%%%%%%%%%%%%%%%

\begin{frame}{CFG Linearization}

\end{frame}

%%%%%%%%%%%%%%%%%%%%%%%%%%%%%%%%%%%%%%%%%%%%%%%%%%%%%%%%%%%%%%%%%%%%%%%%%%%%%%%%

\begin{frame}{Control Flow Conversion: loops}

\end{frame}

%%%%%%%%%%%%%%%%%%%%%%%%%%%%%%%%%%%%%%%%%%%%%%%%%%%%%%%%%%%%%%%%%%%%%%%%%%%%%%%%

\begin{frame}{Finding Loop Live Variables}

\end{frame}

%%%%%%%%%%%%%%%%%%%%%%%%%%%%%%%%%%%%%%%%%%%%%%%%%%%%%%%%%%%%%%%%%%%%%%%%%%%%%%%%

\begin{frame}{Merging Loop Live Variables}

\end{frame}


%\talkpart{3}{Going further}
%%%%%%%%%%%%%%%%%%%%%%%%%%%%%%%%%%%%%%%%%%%%%%%%%%%%%%%%%%%%%%%%%%%%%%%%%%%%%%%%%

\begin{frame}{SIMD Width Detection}

\end{frame}

%%%%%%%%%%%%%%%%%%%%%%%%%%%%%%%%%%%%%%%%%%%%%%%%%%%%%%%%%%%%%%%%%%%%%%%%%%%%%%%%

\begin{frame}{Vectorizing Builtin Function Calls}

% Builtin: the vectorizer has some knowledge of scalar -> vector function mapping

\end{frame}

%%%%%%%%%%%%%%%%%%%%%%%%%%%%%%%%%%%%%%%%%%%%%%%%%%%%%%%%%%%%%%%%%%%%%%%%%%%%%%%%

\begin{frame}{Vectorizing Builtin Functions}

% By vectorizing the builtin's body
% This changes the function signature (return value, some arguments)
% Builtin: the vectorizer has some knowledge of which arguments need packetization
% Need argument placeholders (cloning required)
% Packetized arguments are roots (Uniform Value Analysis)
% Return instructions are leaves (Packetization Stage)

\end{frame}

%%%%%%%%%%%%%%%%%%%%%%%%%%%%%%%%%%%%%%%%%%%%%%%%%%%%%%%%%%%%%%%%%%%%%%%%%%%%%%%%

\begin{frame}{Vectorizing User Functions (No Side-Effects)}

% Similar to builtin functions, but with no knowledge of whether arguments need
% packetization. Need to analyze this for each call site.

\end{frame}

%%%%%%%%%%%%%%%%%%%%%%%%%%%%%%%%%%%%%%%%%%%%%%%%%%%%%%%%%%%%%%%%%%%%%%%%%%%%%%%%

\begin{frame}{Vectorizing User Functions (Side-Effects)}

% Need to pass a mask as an extra argument
% Might be simpler to just inline such functions

\end{frame}

%%%%%%%%%%%%%%%%%%%%%%%%%%%%%%%%%%%%%%%%%%%%%%%%%%%%%%%%%%%%%%%%%%%%%%%%%%%%%%%%

\begin{frame}{Interleaved Memory Optimizations}

\end{frame}

%%%%%%%%%%%%%%%%%%%%%%%%%%%%%%%%%%%%%%%%%%%%%%%%%%%%%%%%%%%%%%%%%%%%%%%%%%%%%%%%

\begin{frame}{SoA to AoS Conversion}

\end{frame}


%%%%%%%%%%%%%%%%%%%%%%%%%%%%%%%%%%%%%%%%%%%%%%%%%%%%%%%%%%%%%%%%%%%%%%%%%%%%%%%%

\section*{Conclusion}

\begin{frame}{Summary}

\begin{itemize}
    \item Should be enough to get started and create a functional vectorizer
    \item Many things were not covered in this talk:
    \begin{itemize}
        \item ...
    \end{itemize}
    \item Introduced resources to go further
\end{itemize}

\end{frame}

%%%%%%%%%%%%%%%%%%%%%%%%%%%%%%%%%%%%%%%%%%%%%%%%%%%%%%%%%%%%%%%%%%%%%%%%%%%%%%%%

\begin{frame}{Thank you!}

\begin{itemize}
    \item Q\&A
    \item ...
    \item Happy vectorization! \hspace{1em} <\hspace{0.15em}\penguin, \penguin, \penguin, \penguin, \penguin, \penguin, \penguin, \penguin\hspace{0.1em}>
\end{itemize}

\vspace{5em}

\end{frame}

%%%%%%%%%%%%%%%%%%%%%%%%%%%%%%%%%%%%%%%%%%%%%%%%%%%%%%%%%%%%%%%%%%%%%%%%%%%%%%%%

\end{document}
