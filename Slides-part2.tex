%%%%%%%%%%%%%%%%%%%%%%%%%%%%%%%%%%%%%%%%%%%%%%%%%%%%%%%%%%%%%%%%%%%%%%%%%%%%%%%%

\talksection{Design}

\begin{frame}{Design}

\begin{itemize}
    \item Pipeline structure
    \begin{itemize}
        \item Function to vectorize is repeatedly transformed by different stages
        \item Each stage consists of one or more IR passes
        \item Most stages require some analysis
        \begin{itemize}
            \item May be run mulitple times as stages invalidate it
        \end{itemize}
    \end{itemize}
    
    \item The LLVM pass manager (both new and original) is a good fit here
    \begin{itemize}
        \item Most passes work at the function level
        \item Access to the module is sometimes needed
    \end{itemize}
    
    \item The original function can be preserved or not
    \begin{itemize}
        \item In-place: vectorize the original function, no need for cloning
        \item Work on cloned function: allow vectorization to fail
    \end{itemize}
\end{itemize}

\end{frame}

%%%%%%%%%%%%%%%%%%%%%%%%%%%%%%%%%%%%%%%%%%%%%%%%%%%%%%%%%%%%%%%%%%%%%%%%%%%%%%%%

\begin{frame}{Stages}

% Stage diagram with stages: preparation, CFG conversion, scalarization, packetization, optimizations

\end{frame}

%%%%%%%%%%%%%%%%%%%%%%%%%%%%%%%%%%%%%%%%%%%%%%%%%%%%%%%%%%%%%%%%%%%%%%%%%%%%%%%%

\begin{frame}{Analyses}

% Two important anlyses: UVA and CFG

\end{frame}

%%%%%%%%%%%%%%%%%%%%%%%%%%%%%%%%%%%%%%%%%%%%%%%%%%%%%%%%%%%%%%%%%%%%%%%%%%%%%%%%

\begin{frame}{Implementation strategy}

\begin{itemize}
    \item Create test kernels
    \begin{itemize}
        \item Start with very simple kernels (e.g. copy buffer, add two buffers)
        \item Gradually add more features (e.g. non-sequential memory accesses, vector instructions, etc)
    \end{itemize}
    
    \item Suggested implementation order
    \begin{itemize}
        \item Preparation and packetization first (required for simplest kernels)
        \item Then easier features: builtins, memory addressing, scalarization, instantiation
        \item More complex features last: control flow, optimizations
    \end{itemize}
\end{itemize}

\end{frame}

%%%%%%%%%%%%%%%%%%%%%%%%%%%%%%%%%%%%%%%%%%%%%%%%%%%%%%%%%%%%%%%%%%%%%%%%%%%%%%%%

\talksection{Stage: Packetization}

\begin{frame}{Packetization Overview}

\begin{itemize}
    \item Stage that does the actual vectorization
    \begin{itemize}
        \item Needs a vectorization factor $N$ (SIMD width)
        \item Calling the transformed function is like calling the original function $N$ times
        \item Mostly straightforward, but usually requires much preparation from previous stages
    \end{itemize}
    
    \item This is done per-instruction, for the whole function
    \begin{itemize}
        \item Instructions that defines a value: define $N$ values, one for each instance
        \item Instruction side effects: perform side effects for each instance
    \end{itemize}
    
    \item Only varying instructions need vectorization
    \begin{itemize}
        \item Uniform instructions can remain scalar, executed once per work-group
        \item Requires uniform value analysis to know which instructions to vectorize
    \end{itemize}
    
\end{itemize}

\end{frame}

%%%%%%%%%%%%%%%%%%%%%%%%%%%%%%%%%%%%%%%%%%%%%%%%%%%%%%%%%%%%%%%%%%%%%%%%%%%%%%%%

\begin{frame}{Uniform Value Analysis}

\begin{itemize}
    \item Finds 'root' values
    \begin{itemize}
        \item Varying value that doesn't have any varying operand
        \item Example: $get_global_id(0)$ has a different value for each isntance
    \end{itemize}
    \item Marks each IR value as 'uniform' or 'varying'
    \begin{itemize}
        \item All values start as 'uniform'
        \item Marking a value as 'varying' causes all users to also be marked 'varying'
        \item Marking is done recursively, starting with roots
    \end{itemize}
\end{itemize}

\end{frame}

%%%%%%%%%%%%%%%%%%%%%%%%%%%%%%%%%%%%%%%%%%%%%%%%%%%%%%%%%%%%%%%%%%%%%%%%%%%%%%%%

\begin{frame}{Uniform Value Analysis Example}

% CL and IR code
% kernel void add_uniform(global int *dst, global int *src, int alpha) {
%     int tid = get_global_id(0);
%     dst[tid] = src[tid] + (alpha + 1);
% }

% Mix of uniform and varying values
\end{frame}

%%%%%%%%%%%%%%%%%%%%%%%%%%%%%%%%%%%%%%%%%%%%%%%%%%%%%%%%%%%%%%%%%%%%%%%%%%%%%%%%

\begin{frame}{Uniform Value Analysis Example}

% IR graph to show values, highlight uniform and varying ones

\end{frame}

%%%%%%%%%%%%%%%%%%%%%%%%%%%%%%%%%%%%%%%%%%%%%%%%%%%%%%%%%%%%%%%%%%%%%%%%%%%%%%%%

\begin{frame}{Packetization Process}

\begin{itemize}
    \item Start at leaves
    \item Packetize operands before instructions
    \item Broadcast uniform operands
    \item Cache packetized values to prevent duplication
\end{itemize}

\end{frame}

%%%%%%%%%%%%%%%%%%%%%%%%%%%%%%%%%%%%%%%%%%%%%%%%%%%%%%%%%%%%%%%%%%%%%%%%%%%%%%%%

\begin{frame}{Packetization Example}

\begin{itemize}
    \item How does it work?
\end{itemize}

\end{frame}

%%%%%%%%%%%%%%%%%%%%%%%%%%%%%%%%%%%%%%%%%%%%%%%%%%%%%%%%%%%%%%%%%%%%%%%%%%%%%%%%

\begin{frame}{Packetization: Phi nodes}

\begin{itemize}
    \item How to handle this special case
\end{itemize}

\end{frame}

%%%%%%%%%%%%%%%%%%%%%%%%%%%%%%%%%%%%%%%%%%%%%%%%%%%%%%%%%%%%%%%%%%%%%%%%%%%%%%%%

\begin{frame}{Memory addressing}

\begin{itemize}
    \item Uniform address -> scalar loads and stores
    \item Constant stride =1 -> vector laods and stores
    \item Constant stride >1 -> interleaved loads and stores
    \item Other -> gather loads, scatter stores
\end{itemize}

% Generate calls to internal builtins
% Internal builtins can be implemented for each target as supported

\end{frame}

%%%%%%%%%%%%%%%%%%%%%%%%%%%%%%%%%%%%%%%%%%%%%%%%%%%%%%%%%%%%%%%%%%%%%%%%%%%%%%%%

\talksection{Stage: Scalarization}

\begin{frame}{Scalarization Overview}

\begin{itemize}
    \item What does it do?
    \item Requires scalarization analysis
\end{itemize}

\end{frame}

%%%%%%%%%%%%%%%%%%%%%%%%%%%%%%%%%%%%%%%%%%%%%%%%%%%%%%%%%%%%%%%%%%%%%%%%%%%%%%%%

\begin{frame}{Scalarization Analysis}

\begin{itemize}
    \item Looks for vector instructions
    \begin{itemize}
        \item Leaves that define vector values, vector stores
        \item Vector extractions
        \item Vector -> scalar bitcasts
    \end{itemize}
    
\end{itemize}

\end{frame}

%%%%%%%%%%%%%%%%%%%%%%%%%%%%%%%%%%%%%%%%%%%%%%%%%%%%%%%%%%%%%%%%%%%%%%%%%%%%%%%%

\begin{frame}{Scalarization Process}


\end{frame}

%%%%%%%%%%%%%%%%%%%%%%%%%%%%%%%%%%%%%%%%%%%%%%%%%%%%%%%%%%%%%%%%%%%%%%%%%%%%%%%%

\begin{frame}{Scalarization Example}


\end{frame}

%%%%%%%%%%%%%%%%%%%%%%%%%%%%%%%%%%%%%%%%%%%%%%%%%%%%%%%%%%%%%%%%%%%%%%%%%%%%%%%%

\talksection{Stage: Control Flow Conversion}

\begin{frame}{Control Flow Conversion Overview}

\begin{itemize}
    \item What does it do?
    \item Why is it needed?
\end{itemize}

\end{frame}

%%%%%%%%%%%%%%%%%%%%%%%%%%%%%%%%%%%%%%%%%%%%%%%%%%%%%%%%%%%%%%%%%%%%%%%%%%%%%%%%

\begin{frame}{Control Flow Conversion: if}

\end{frame}

%%%%%%%%%%%%%%%%%%%%%%%%%%%%%%%%%%%%%%%%%%%%%%%%%%%%%%%%%%%%%%%%%%%%%%%%%%%%%%%%

\begin{frame}{Mask Generation}

\end{frame}

%%%%%%%%%%%%%%%%%%%%%%%%%%%%%%%%%%%%%%%%%%%%%%%%%%%%%%%%%%%%%%%%%%%%%%%%%%%%%%%%

\begin{frame}{Applying Masks}

\end{frame}

%%%%%%%%%%%%%%%%%%%%%%%%%%%%%%%%%%%%%%%%%%%%%%%%%%%%%%%%%%%%%%%%%%%%%%%%%%%%%%%%

\begin{frame}{Masked Memory Operations}

\end{frame}

%%%%%%%%%%%%%%%%%%%%%%%%%%%%%%%%%%%%%%%%%%%%%%%%%%%%%%%%%%%%%%%%%%%%%%%%%%%%%%%%

\begin{frame}{Phi Conversion}

\end{frame}

%%%%%%%%%%%%%%%%%%%%%%%%%%%%%%%%%%%%%%%%%%%%%%%%%%%%%%%%%%%%%%%%%%%%%%%%%%%%%%%%

\begin{frame}{CFG Linearization}

\end{frame}

%%%%%%%%%%%%%%%%%%%%%%%%%%%%%%%%%%%%%%%%%%%%%%%%%%%%%%%%%%%%%%%%%%%%%%%%%%%%%%%%

\begin{frame}{Control Flow Conversion: loops}

\end{frame}

%%%%%%%%%%%%%%%%%%%%%%%%%%%%%%%%%%%%%%%%%%%%%%%%%%%%%%%%%%%%%%%%%%%%%%%%%%%%%%%%

\begin{frame}{Finding Loop Live Variables}

\end{frame}

%%%%%%%%%%%%%%%%%%%%%%%%%%%%%%%%%%%%%%%%%%%%%%%%%%%%%%%%%%%%%%%%%%%%%%%%%%%%%%%%

\begin{frame}{Merging Loop Live Variables}

\end{frame}

